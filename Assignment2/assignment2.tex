\documentclass{article}
\setlength{\topmargin}{5in}
\usepackage[a4paper, total={6in, 8in}]{geometry}
\usepackage{layout}
\usepackage{changepage,titlesec}
\usepackage{listings}
\usepackage{color}


\title{\vspace{-2cm}{Assignment 1: COP 290: Design Document}}
\author{Aniket Khandelwal(2013CS10209),
Anupam Khandelwal(2013CS10212),\\
Ronak Khandelwal(2013CS50295)}
\date{\today}
\lstset{
frame=tb, % draw a frame at the top and bottom of the code block
tabsize=4, % tab space width
showstringspaces=false, % don't mark spaces in strings
numbers=left, % display line numbers on the left
commentstyle=\color{green}, % comment color
keywordstyle=\color{cyan}, % keyword color
stringstyle=\color{} % string color
}

\begin{document}

\maketitle

\tableofcontents
\pagebreak

\section{Abstract}
We will build an online file management system \"MyDropBox\" in which a server machine maintains the files of multiple users. The user will use  a simple desktop application to 
login into the system and can access his/her files after authentication. We will have two modes:
\begin{itemize}
\item Offline mode: In this we will transfer files from one folder to another within a system after the user login in the dekstop application
\item Online mode: We will extend the offline mode by including a server so that the client can store and access the data which he has stored on the server
\end{itemize}
\section{Overall Design}
\begin{itemize}
\item We will start the assignment by building different subcomponents like GUI, server and client codes, sharing and handling of files and then we will do the testing of these subcomponents
\item When we are done with testing of the subcomponents, we will link them together and get the offline mode working
\item After the offline mode gets working, we will include the part of the server and get the online mode to run. We will ensure that all the data handling and data transfer is done in a secure way
\end{itemize}

\section{Details of the Sub Components}
\subsection{User Interface}
We will use QT Creator for designing the client Graphic User Interface. We will have different pages which will be as follows:
\begin{itemize}
\item First page will ask the user whether he is a new user or an existing one and the user will have to sign up if he is a new user otherwise he will sign in to the login page
\item Second page is the login page. In this user will have to enter the username and password for authentication and if both matches with the information given by the user at the time of signup, then he will be able to login into his account
\item If the user is a new user, he will have to sign up and fill the required information to create his account on \"MyDropBox\" and then he can sign in to access his account
\item If the user has forgot the password, then we will ask him a security question, if the username is correct, which the user will give us at the time of creating his account. Then we will allow him to set new username and password
\item In the final page, the user will be able to see files on the system and can do various operations on the files like deleting the files, adding new files and then finally he needs to press a sync button to store the file on the server. Also he can share the files with the other users while setting the appropriate permissions at the the same time
\end{itemize}
\subsection{Server part}
TO BE DONE BY ANUPAM
Write about the classes in points
\subsection{Networking}
\subsubsection{Sharing}
\begin{itemize}
\item We will use FTP(File transfer protocol) for creating connection between server and the client
\item Then after setting up the connection, the user can share files with the other users. Also he can set the permissions to give certain level of access to the other user. In the backend, we will have the code which will assign the file and the permissions associated to it to the user to whom it has been shared
\end{itemize}
\subsubsection{Security}
\begin{itemize}
\item For secure connection between the client and the server, we will use OpenSSL.
\item WRITE ABOUT WHAT WILL YOU DO IT OPENSSL
\item \textbf{TO BE IMPROVED}We will also ensure that user can access the data after he has logged into the account. This will ensure that even if multiple user uses the same machine, one user can't access other user's data
\end{itemize}
\subsection{Syncing of the files}
\begin{itemize}
\item The user will be allowed to sync the files between the server and his own folder. This will ensure that he can access the files on any machine after logging into the application
\item Our code at the backend will ensure that the syncing process goes smooth. If there are two versions of the same file, then we will check whether one of the two files (the file on the server or the file on the system) is newer. Then we will warn the user about the file which is newer and he can then take appropriate action to deal with the file
\end{itemize}
\subsection{File Handling}
\subsubsection{Minimizing Data Usage}
TO BE EDITED
\subsubsection{Handling Changes in the files}
TO BE EDITED
\subsection{Persistent Storage}
TO BE EDITED
\section{Testing of the program}
\subsection{Testing of Sub Components}
\subsubsection{User Interface Testing}
\begin{itemize}
\item We will first test each page of the user interface by using different possible inputs
\item After this we will do the testing of the whole interface by applying different possible combinations of input and seeing that the transitions between the pages and other things work correctly
\end{itemize}
\subsubsection{Testing of sharing of files TO BE EDITED}
\begin{itemize}
\item We will first do the unit testing of the classes that are needed to share the file. For this, we will test the constructors of each class, get and set functions of these classes.
\item When we are done with the testing of the classes, we will test it by sharing files of different sizes first between the two folders in the same system(Offline mode) and then we will share the files between two systems and check whether the files are send correctly by running them and checking their sizes
\end{itemize}
\subsubsection{Testing of syncing of files}
\begin{itemize}
\item We will first do the testing of the classes in the same as we will do it in testing of sharing of files
\item Then we will check the syncing in the offline mode by adding new files, deleting files and adding different version of the same file. We will test whether we are able to add and delete files in the other folder properly and also whether we are able to tell which of the files having different version is newer and whether we are to appropriately deal with this case
\item After this, we will test the online syncing of the files on the same parameters which we have used in the offline mode
\end{itemize}
\subsubsection{Testing of file handling}
TO BE DONE
\subsubsection{Testing of persistent storage}
TO BE DONE
\subsection{Overall testing}
After doing the unit testing of each subcomponent, we will link all the subcomponents and then we will test the whole program starting from user interface, setting connection between server and client, sharing data securely,with other user, syncing the data using the parameters which has been used during unit testing. Also we will check whether the files are handled in a proper way i.e. the data usage is optimized and the changes in the file are handled in a proper way or not. We will also check whether the user is able to access the files from the local folder without downloading the files even after the connection is lost.
\section{Interaction between Sub Components}
TO BE EDITED
\section{Extra Features}
TO BE EDITED
\end{document}